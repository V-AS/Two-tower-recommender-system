\documentclass[12pt, titlepage]{article}

\usepackage{booktabs}
\usepackage{tabularx}
\usepackage{hyperref}
\hypersetup{
    colorlinks,
    citecolor=black,
    filecolor=black,
    linkcolor=red,
    urlcolor=blue
}
\usepackage[round]{natbib}

\input{../Comments}
\input{../Common}

\begin{document}

\title{Verification and Validation Report: \progname} 
\author{\authname}
\date{\today}
	
\maketitle

\pagenumbering{roman}

\section{Revision History}

\begin{tabularx}{\textwidth}{p{3cm}p{2cm}X}
\toprule {\bf Date} & {\bf Version} & {\bf Notes}\\
\midrule
Apr. 2 2025 & 1.0 & First draft\\
\bottomrule
\end{tabularx}
\nocite{*}
~\newpage

\section{Symbols, Abbreviations and Acronyms}

\renewcommand{\arraystretch}{1.2}
\begin{tabular}{l l} 
  \toprule		
  \textbf{symbol} & \textbf{description}\\
  \midrule 
  T & Test\\
  \bottomrule
\end{tabular}\\

\wss{symbols, abbreviations or acronyms -- you can reference the SRS tables if needed}

\newpage

\tableofcontents

\listoftables %if appropriate

\listoffigures %if appropriate

\newpage

\pagenumbering{arabic}

This document includes the results of \href{https://github.com/V-AS/Two-tower-recommender-system/blob/main/docs/VnVPlan/VnVPlan.pdf}{VnV plan}.

\section{Functional Requirements Evaluation}

The functional requirements are tested using both system tests and unit tests.

\subsection{Dataset}

To ensure the project receives a valid dataset for training:

Manually, a Jupyter Notebook (data\_preprocessing.ipynb) is used to merge the raw data (three CSV files) into a single CSV file. 

Then, the automated system test (test\_data\_validation.py) runs each time the dataset is updated before training starts to ensure that the input CSV file contains all the required columns.

Additionally, an automated unit test (test\_data\_processing.py) runs on each push to verify the correctness of the data processor, which generates new features for users and items.

All tests have passed successfully.

\subsection{Model Training Convergence}
The system test (test\_model\_convergence.py) checks whether the training loss decreases as training progresses.

This test has passed successfully.

\subsection{Model storage}

The system test (test\_model\_storage.py) runs each time model training is completed. The test ensures that the trained model and computed embeddings are correctly stored in the `output' folder.

This test has passed successfully.

\subsection{Embedding generation}

The unit test `test\_embedding\_generation.py' will check the correctness of embedding generation. This test has passed successfully.

\section{Nonfunctional Requirements Evaluation}

\subsection{Usability}
		
\subsection{Reliability}

\subsection{Protability}
	

\section{Unit Testing}

\section{Changes Due to Testing}

\wss{This section should highlight how feedback from the users and from 
the supervisor (when one exists) shaped the final product.  In particular 
the feedback from the Rev 0 demo to the supervisor (or to potential users) 
should be highlighted.}

\section{Automated Testing}
		
\section{Trace to Requirements}
		
\section{Trace to Modules}		

\section{Code Coverage Metrics}

\bibliographystyle{plainnat}
\bibliography{../../refs/References}

\newpage{}
\section*{Appendix --- Reflection}

The information in this section will be used to evaluate the team members on the
graduate attribute of Reflection. 


\begin{enumerate}
  \item What went well while writing this deliverable? 
  \item What pain points did you experience during this deliverable, and how
    did you resolve them?
  \item Which parts of this document stemmed from speaking to your client(s) or
  a proxy (e.g. your peers)? Which ones were not, and why?
  \item In what ways was the Verification and Validation (VnV) Plan different
  from the activities that were actually conducted for VnV?  If there were
  differences, what changes required the modification in the plan?  Why did
  these changes occur?  Would you be able to anticipate these changes in future
  projects?  If there weren't any differences, how was your team able to clearly
  predict a feasible amount of effort and the right tasks needed to build the
  evidence that demonstrates the required quality?  (It is expected that most
  teams will have had to deviate from their original VnV Plan.)
\end{enumerate}

\end{document}