\documentclass[12pt, titlepage]{article}

\usepackage{amsmath, mathtools}

\usepackage[round]{natbib}
\usepackage{amsfonts}
\usepackage{amssymb}
\usepackage{graphicx}
\usepackage{colortbl}
\usepackage{xr}
\usepackage{hyperref}
\usepackage{longtable}
\usepackage{xfrac}
\usepackage{tabularx}
\usepackage{float}
\usepackage{siunitx}
\usepackage{booktabs}
\usepackage{multirow}
\usepackage[section]{placeins}
\usepackage{caption}
\usepackage{fullpage}

\hypersetup{
bookmarks=true,     % show bookmarks bar?
colorlinks=true,       % false: boxed links; true: colored links
linkcolor=red,          % color of internal links (change box color with linkbordercolor)
citecolor=blue,      % color of links to bibliography
filecolor=magenta,  % color of file links
urlcolor=cyan          % color of external links
}

\usepackage{array}

\externaldocument{../../SRS/SRS}

\input{../../Comments}
\input{../../Common}

\begin{document}

\title{Module Interface Specification for \progname{}}

\author{\authname}

\date{\today}

\maketitle

\pagenumbering{roman}

\section{Revision History}

\begin{tabularx}{\textwidth}{p{3cm}p{2cm}X}
\toprule {\bf Date} & {\bf Version} & {\bf Notes}\\
\midrule
March 2 2025 & 1.0 & First Draft\\
\bottomrule
\end{tabularx}

~\newpage

\section{Symbols, Abbreviations and Acronyms}

See SRS Documentation at \url{https://github.com/V-AS/Two-tower-recommender-system/blob/main/docs/SRS/SRS.pdf}
% Also add any additional symbols, abbreviations or acronyms

\newpage

\tableofcontents

\newpage

\pagenumbering{arabic}

\section{Introduction}

The following document details the Module Interface Specifications for
\progname

Complementary documents include the System Requirement Specifications
and Module Guide.  The full documentation and implementation can be
found at \url{https://github.com/V-AS/Two-tower-recommender-system}

\section{Notation}

\wss{You should describe your notation.  You can use what is below as
  a starting point.}

The structure of the MIS for modules comes from \citet{HoffmanAndStrooper1995},
with the addition that template modules have been adapted from
\cite{GhezziEtAl2003}.

The following table summarizes the primitive data types used by \progname. 

\begin{center}
\renewcommand{\arraystretch}{1.2}
\noindent 
\begin{tabular}{l l p{7.5cm}} 
\toprule 
\textbf{Data Type} & \textbf{Notation} & \textbf{Description}\\ 
\midrule
character & char & A sequence of characters\\
Array & $[T]$ & A sequence of elements of type $T$\\
Matrix &$[T]^{m\times n}$ & A 2D array of type T with m rows and n columns\\
Boolean & $\mathbb{B}$ & True or False value\\
Integer & $\mathbb{Z}$ & A number without a fractional component in (-$\infty$, $\infty$) \\
real & $\mathbb{R}$ & Any number in (-$\infty$, $\infty$)\\
\bottomrule
\end{tabular} 
\end{center}

\noindent
\progname \ uses functions, which
are defined by the data types of their inputs and outputs. Local functions are
described by giving their type signature followed by their specification.

The specification also uses derived data types:
\begin{itemize}
  \item Embedding: A vector of real numbers
  \item Tensors: Multi-dimensional arrays
  \item User: A type representing user features
  \item Item: A type representing item features
\end{itemize}


\section{Module Decomposition}

The following table is taken directly from the Module Guide document for this project.

\begin{table}[h!]
\centering
\begin{tabular}{p{0.3\textwidth} p{0.6\textwidth}}
\toprule
\textbf{Level 1} & \textbf{Level 2}\\
\midrule

{Hardware-Hiding Module} & ~ \\
\midrule

\multirow{7}{0.3\textwidth}{Behaviour-Hiding Module} & Data Processing Module\\
& Model Training Module\\
& Embedding Generation Module\\
& Recommendation Module\\
\midrule

\multirow{3}{0.3\textwidth}{Software Decision Module} & {Neural Network Architecture Module}\\
& ANN Search Module\\
& Vector Operations Module\\
\bottomrule

\end{tabular}
\caption{Module Hierarchy}
\label{TblMH}
\end{table}


\newpage
~\newpage

\section{MIS of Hardware-Hiding Module} \label{ModuleHH}

\subsection{Module}

SystemInterface

\subsection{Uses}
None

\subsection{Syntax}

\subsubsection{Exported Constants}
None
\subsubsection{Exported Access Programs}

\begin{center}
\begin{tabular}{p{2cm} p{4cm} p{4cm} p{2cm}}
\hline
\textbf{Name} & \textbf{In} & \textbf{Out} & \textbf{Exceptions} \\
\hline
save\_model & model: Model, path: String & success: $\mathbb{B}$ & IOError \\
\hline
load\_model & path: String & model: Model & IOError,

FormatError \\
\hline
save\_emds & embeddings: [Embedding],

path: String & success: $\mathbb{B}$ & IOError \\
\hline
load\_emds & path: String & embeddings: [Embedding] & IOError \\
\hline
\end{tabular}
\end{center}

\subsection{Semantics}

\subsubsection{State Variables}
None

\subsubsection{Environment Variables}

FileSystem: The file system where models and embeddings are stored

\subsubsection{Assumptions}

\begin{itemize}
  \item The file system is accessible and has sufficient space
  \item The paths provided are valid
\end{itemize}

\subsubsection{Access Routine Semantics}

\noindent save\_model(model, path):
\begin{itemize}
\item output:  success = true if operation succeeds
\item exception: IOError if file cannot be written
\end{itemize}

\noindent load\_model(path):
\begin{itemize}
\item output: model
\item exception: IOError if file cannot be read, FormatError if file format is invalid
\end{itemize}

\noindent save\_embeddings(embeddings, path):
\begin{itemize}
\item output: success = true if operation succeeds
\item exception: IOError if file cannot be written
\end{itemize}

\noindent oad\_embeddings(path):
\begin{itemize}
\item output: embeddings
\item exception: IOError if file cannot be read, FormatError if file format is invalid
\end{itemize}


\section{MIS of Data Processing Module} \label{ModuleDP}

\subsection{Module}

DataProcessor

\subsection{Uses}
SystemInterface

\subsection{Syntax}

\subsubsection{Exported Constants}
None
\subsubsection{Exported Access Programs}

\begin{center}
\begin{tabular}{p{2cm} p{4cm} p{4cm} p{2cm}}
\hline
\textbf{Name} & \textbf{In} & \textbf{Out} & \textbf{Exceptions} \\
\hline
load\_data & path: String & data: DataSet & IOError,

FormatError\\
\hline
validate\_data & data: DataSet & is\_valid: $\mathbb{B}$ & -\\
\hline
prep\_data & data: DataSet & processed\_dataset & - \\
\hline
split\_data & data: DataSet,

train\_ratio: $\mathbb{R}$ & train\_data: DataSet,

test\_data: DataSet & IOError \\
\hline
\end{tabular}
\end{center}

\subsection{Semantics}

\subsubsection{State Variables}
None

\subsubsection{Environment Variables}

None

\subsubsection{Assumptions}

\begin{itemize}
  \item Input data follows the expected schema
\end{itemize}

\subsubsection{Access Routine Semantics}

\noindent load\_data(path):
\begin{itemize}
\item data = parsed data from file at path
\item exception: IOError if file cannot be read, FormatError if file format is invalid
\end{itemize}

\noindent validate\_data(data)::
\begin{itemize}
\item output: is\_valid = true if data meets all validation criteria
\end{itemize}

\noindent preprocess\_data(data):
\begin{itemize}
\item output: processed\_data = dataset after applying preprocessing transformations
\end{itemize}

\noindent split\_data(data, train\_ratio):
\begin{itemize}
\item output: (train\_data, test\_data) where:
\begin{itemize}
  \item train\_data = subset of data for training (size $\approx$ train\_ratio $* |data|$)
  \item test\_data = subset of data for training (size $\approx$ (1-train\_ratio) $* |data|$)
\end{itemize}
\item exception: ValueError if train\_ratio is not in $(0,1)$
\end{itemize}

\section{MIS of Model Training Module} \label{ModuleMT}

\subsection{Module}

ModelTrainer

\subsection{Uses}
DataProcessor, NeuralNetworkArchitecture, VectorOperations

\subsection{Syntax}

\subsubsection{Exported Constants}
DEFAULT\_LEARNING\_RATE = 0.01\\
DEFAULT\_BATCH\_SIZE = 128\\
DEFAULT\_REGULARIZATION = 0.01
\subsubsection{Exported Access Programs}

\begin{center}
\begin{tabular}{p{2cm} p{4cm} p{4cm} p{2cm}}
\hline
\textbf{Name} & \textbf{In} & \textbf{Out} & \textbf{Exceptions} \\
\hline
initialize & config: TrainingConfig & - & ValueError \\
\hline
train & train\_data: DataSet,

epochs: $\mathbb{Z}$

& model: Model & IOError,

FormatError \\
\hline
save\_emds & embeddings: [Embedding],

path: String & success: $\mathbb{B}$ & IOError \\
\hline
load\_emds & path: String & embeddings: [Embedding] & IOError \\
\hline
\end{tabular}
\end{center}

\subsection{Semantics}

\subsubsection{State Variables}
None

\subsubsection{Environment Variables}

FileSystem: The file system where models and embeddings are stored

\subsubsection{Assumptions}

\begin{itemize}
  \item The file system is accessible and has sufficient space
  \item The paths provided are valid
\end{itemize}

\subsubsection{Access Routine Semantics}

\noindent save\_model(model, path):
\begin{itemize}
\item output:  success = true if operation succeeds
\item exception: IOError if file cannot be written
\end{itemize}

\noindent load\_model(path):
\begin{itemize}
\item output: model
\item exception: IOError if file cannot be read, FormatError if file format is invalid
\end{itemize}

\noindent save\_embeddings(embeddings, path):
\begin{itemize}
\item output: success = true if operation succeeds
\item exception: IOError if file cannot be written
\end{itemize}

\noindent oad\_embeddings(path):
\begin{itemize}
\item output: embeddings
\item exception: IOError if file cannot be read, FormatError if file format is invalid
\end{itemize}


\section{MIS of Hardware-Hiding Module} \label{ModuleHH}

\subsection{Module}

SystemInterface

\subsection{Uses}
None

\subsection{Syntax}

\subsubsection{Exported Constants}
None
\subsubsection{Exported Access Programs}

\begin{center}
\begin{tabular}{p{2cm} p{4cm} p{4cm} p{2cm}}
\hline
\textbf{Name} & \textbf{In} & \textbf{Out} & \textbf{Exceptions} \\
\hline
save\_model & model: Model, path: String & success: $\mathbb{B}$ & IOError \\
\hline
load\_model & path: String & model: Model & IOError,

FormatError \\
\hline
save\_emds & embeddings: [Embedding],

path: String & success: $\mathbb{B}$ & IOError \\
\hline
load\_emds & path: String & embeddings: [Embedding] & IOError \\
\hline
\end{tabular}
\end{center}

\subsection{Semantics}

\subsubsection{State Variables}
None

\subsubsection{Environment Variables}

FileSystem: The file system where models and embeddings are stored

\subsubsection{Assumptions}

\begin{itemize}
  \item The file system is accessible and has sufficient space
  \item The paths provided are valid
\end{itemize}

\subsubsection{Access Routine Semantics}

\noindent save\_model(model, path):
\begin{itemize}
\item output:  success = true if operation succeeds
\item exception: IOError if file cannot be written
\end{itemize}

\noindent load\_model(path):
\begin{itemize}
\item output: model
\item exception: IOError if file cannot be read, FormatError if file format is invalid
\end{itemize}

\noindent save\_embeddings(embeddings, path):
\begin{itemize}
\item output: success = true if operation succeeds
\item exception: IOError if file cannot be written
\end{itemize}

\noindent oad\_embeddings(path):
\begin{itemize}
\item output: embeddings
\item exception: IOError if file cannot be read, FormatError if file format is invalid
\end{itemize}


\section{MIS of Hardware-Hiding Module} \label{ModuleHH}

\subsection{Module}

SystemInterface

\subsection{Uses}
None

\subsection{Syntax}

\subsubsection{Exported Constants}
None
\subsubsection{Exported Access Programs}

\begin{center}
\begin{tabular}{p{2cm} p{4cm} p{4cm} p{2cm}}
\hline
\textbf{Name} & \textbf{In} & \textbf{Out} & \textbf{Exceptions} \\
\hline
save\_model & model: Model, path: String & success: $\mathbb{B}$ & IOError \\
\hline
load\_model & path: String & model: Model & IOError,

FormatError \\
\hline
save\_emds & embeddings: [Embedding],

path: String & success: $\mathbb{B}$ & IOError \\
\hline
load\_emds & path: String & embeddings: [Embedding] & IOError \\
\hline
\end{tabular}
\end{center}

\subsection{Semantics}

\subsubsection{State Variables}
None

\subsubsection{Environment Variables}

FileSystem: The file system where models and embeddings are stored

\subsubsection{Assumptions}

\begin{itemize}
  \item The file system is accessible and has sufficient space
  \item The paths provided are valid
\end{itemize}

\subsubsection{Access Routine Semantics}

\noindent save\_model(model, path):
\begin{itemize}
\item output:  success = true if operation succeeds
\item exception: IOError if file cannot be written
\end{itemize}

\noindent load\_model(path):
\begin{itemize}
\item output: model
\item exception: IOError if file cannot be read, FormatError if file format is invalid
\end{itemize}

\noindent save\_embeddings(embeddings, path):
\begin{itemize}
\item output: success = true if operation succeeds
\item exception: IOError if file cannot be written
\end{itemize}

\noindent oad\_embeddings(path):
\begin{itemize}
\item output: embeddings
\item exception: IOError if file cannot be read, FormatError if file format is invalid
\end{itemize}


\section{MIS of Hardware-Hiding Module} \label{ModuleHH}

\subsection{Module}

SystemInterface

\subsection{Uses}
None

\subsection{Syntax}

\subsubsection{Exported Constants}
None
\subsubsection{Exported Access Programs}

\begin{center}
\begin{tabular}{p{2cm} p{4cm} p{4cm} p{2cm}}
\hline
\textbf{Name} & \textbf{In} & \textbf{Out} & \textbf{Exceptions} \\
\hline
save\_model & model: Model, path: String & success: $\mathbb{B}$ & IOError \\
\hline
load\_model & path: String & model: Model & IOError,

FormatError \\
\hline
save\_emds & embeddings: [Embedding],

path: String & success: $\mathbb{B}$ & IOError \\
\hline
load\_emds & path: String & embeddings: [Embedding] & IOError \\
\hline
\end{tabular}
\end{center}

\subsection{Semantics}

\subsubsection{State Variables}
None

\subsubsection{Environment Variables}

FileSystem: The file system where models and embeddings are stored

\subsubsection{Assumptions}

\begin{itemize}
  \item The file system is accessible and has sufficient space
  \item The paths provided are valid
\end{itemize}

\subsubsection{Access Routine Semantics}

\noindent save\_model(model, path):
\begin{itemize}
\item output:  success = true if operation succeeds
\item exception: IOError if file cannot be written
\end{itemize}

\noindent load\_model(path):
\begin{itemize}
\item output: model
\item exception: IOError if file cannot be read, FormatError if file format is invalid
\end{itemize}

\noindent save\_embeddings(embeddings, path):
\begin{itemize}
\item output: success = true if operation succeeds
\item exception: IOError if file cannot be written
\end{itemize}

\noindent oad\_embeddings(path):
\begin{itemize}
\item output: embeddings
\item exception: IOError if file cannot be read, FormatError if file format is invalid
\end{itemize}



\subsubsection{Local Functions}

\wss{As appropriate} \wss{These functions are for the purpose of specification.
  They are not necessarily something that is going to be implemented
  explicitly.  Even if they are implemented, they are not exported; they only
  have local scope.}

\newpage

\bibliographystyle {plainnat}
\bibliography {../../../refs/References}

\newpage

\section{Appendix} \label{Appendix}

\wss{Extra information if required}

\newpage{}

\section*{Appendix --- Reflection}

\wss{Not required for CAS 741 projects}

The information in this section will be used to evaluate the team members on the
graduate attribute of Problem Analysis and Design.

\begin{enumerate}
  \item What went well while writing this deliverable? 
  \item What pain points did you experience during this deliverable, and how
    did you resolve them?
  \item Which of your design decisions stemmed from speaking to your client(s)
  or a proxy (e.g. your peers, stakeholders, potential users)? For those that
  were not, why, and where did they come from?
  \item While creating the design doc, what parts of your other documents (e.g.
  requirements, hazard analysis, etc), it any, needed to be changed, and why?
  \item What are the limitations of your solution?  Put another way, given
  unlimited resources, what could you do to make the project better? (LO\_ProbSolutions)
  \item Give a brief overview of other design solutions you considered.  What
  are the benefits and tradeoffs of those other designs compared with the chosen
  design?  From all the potential options, why did you select the documented design?
  (LO\_Explores)
\end{enumerate}


\end{document}