\documentclass[12pt, titlepage]{article}

\usepackage{amsmath, mathtools}

\usepackage[round]{natbib}
\usepackage{amsfonts}
\usepackage{amssymb}
\usepackage{graphicx}
\usepackage{colortbl}
\usepackage{xr}
\usepackage{hyperref}
\usepackage{longtable}
\usepackage{xfrac}
\usepackage{tabularx}
\usepackage{float}
\usepackage{siunitx}
\usepackage{booktabs}
\usepackage{multirow}
\usepackage[section]{placeins}
\usepackage{caption}
\usepackage{fullpage}

\hypersetup{
bookmarks=true,     % show bookmarks bar?
colorlinks=true,       % false: boxed links; true: colored links
linkcolor=red,          % color of internal links (change box color with linkbordercolor)
citecolor=blue,      % color of links to bibliography
filecolor=magenta,  % color of file links
urlcolor=cyan          % color of external links
}

\usepackage{array}

\externaldocument{../../SRS/SRS}

\input{../../Comments}
\input{../../Common}

\begin{document}

\title{Module Interface Specification for \progname{}}

\author{\authname}

\date{\today}

\maketitle

\pagenumbering{roman}

\section{Revision History}

\begin{tabularx}{\textwidth}{p{3cm}p{2cm}X}
\toprule {\bf Date} & {\bf Version} & {\bf Notes}\\
\midrule
Mar. 2 2025 & 1.0 & First Draft\\
Apr. 11 2025 & 2.0 & Revision 1\\
\bottomrule
\end{tabularx}

~\newpage

\section{Symbols, Abbreviations and Acronyms}

See SRS Documentation at \url{https://github.com/V-AS/Two-tower-recommender-system/blob/main/docs/SRS/SRS.pdf}
% Also add any additional symbols, abbreviations or acronyms

\newpage

\tableofcontents

\newpage

\pagenumbering{arabic}

\section{Introduction}

The following document details the Module Interface Specifications for
\progname, a two-tower recommendation system. The system leverages deep learning to create both the user tower and item tower, mapping inputs to a shared embedding space. Then, an effective algorithm is used to select a large number of candidate items. Finally, the dot product is applied for a refined ranking of the candidate items, returning the final recommendations accordingly.

Complementary documents include the System Requirement Specifications
and Module Guide.  The full documentation and implementation can be
found at \url{https://github.com/V-AS/Two-tower-recommender-system}

\section{Notation}

The structure of the MIS for modules comes from \citet{HoffmanAndStrooper1995},
with the addition that template modules have been adapted from
\cite{GhezziEtAl2003}.

The following table summarizes the primitive data types used by \progname. 

\begin{center}
\renewcommand{\arraystretch}{1.2}
\noindent 
\begin{tabular}{l l p{7.5cm}} 
\toprule 
\textbf{Data Type} & \textbf{Notation} & \textbf{Description}\\ 
\midrule
Character & char & A single character\\
String & string & A sequence of characters representing text\\
Array & $[T]$ & A sequence of elements of type $T$\\
Dictionary/Map & dict & A Python dictionary\\
Vector & $\mathbb{R}^n$ & An ordered collection of n real numbers\\
Matrix &$[T]^{m\times n}$ & A 2D array of type T with m rows and n columns\\
Boolean & $\mathbb{B}$ & True or False value\\
Integer & $\mathbb{Z}$ & A number without a fractional component in (-$\infty$, $\infty$) \\
Real & $\mathbb{R}$ & Any number in (-$\infty$, $\infty$)\\
Tuple & $(T_1, T_2, ...)$ & An ordered collection of elements with possibly different types\\
\bottomrule
\end{tabular} 
\end{center}

\noindent
\progname \ uses functions, which
are defined by the data types of their inputs and outputs. Local functions are
described by giving their type signature followed by their specification.

The specification also uses derived data types:
\begin{itemize}
  \item \textbf{Embedding}: A vector of real numbers representing learned features.
  \item \textbf{Tensor}: A multi-dimensional array, used for numerical computations.
  \item \textbf{DataFrame}: A two-dimensional array-like structure with labeled axes (rows and columns), typically used for storing tabular data.
  \item \textbf{Dataset}: A collection of data used for training or testing machine learning models.
  \item \textbf{User Feature}: A dictionary where the key is a string and the value is the corresponding feature value for the user.
  \item \textbf{Item Feature}: A dictionary where the key is a string and the value is the corresponding feature value for the item.
  \item \textbf{Model}: A neural network used for learning user and item representations.
  \item \textbf{ANNIndex}: A data structure used for approximate nearest neighbor search in embedding space.
  \item \textbf{EvaluationMetrics}: A dictionary mapping evaluation metric names (strings) to their computed values (e.g., accuracy, RMSE).
  \item \textbf{TrainingConfig}: A dictionary containing configuration parameters for model training.
  \item \textbf{FormatError}: An exception raised when a file has an incorrect format.
  \item \textbf{DimensionMismatchError}: An exception raised when vectors of incompatible dimensions are used in operations.
\end{itemize}

\section{Module Decomposition}

The following table is taken directly from the Module Guide document for this project.

\begin{table}[h!]
\centering
\begin{tabular}{p{0.3\textwidth} p{0.6\textwidth}}
\toprule
\textbf{Level 1} & \textbf{Level 2}\\
\midrule

{Hardware-Hiding Module} & ~ \\
\midrule

\multirow{7}{0.3\textwidth}{Behaviour-Hiding Module} & System Interface Module\\
& Data Processing Module\\
& Model Training Module\\
& Embedding Generation Module\\
& Recommendation Module\\
\midrule

\multirow{3}{0.3\textwidth}{Software Decision Module} & {Neural Network Architecture Module}\\
& ANN Search Module\\
& Vector Operations Module\\
\bottomrule

\end{tabular}
\caption{Module Hierarchy}
\label{TblMH}
\end{table}


\newpage


\section{MIS of System Interface Module} \label{ModuleHH}

\subsection{Module}

SystemInterface

\subsection{Uses}
None

\subsection{Syntax}

\subsubsection{Exported Constants}
None
\subsubsection{Exported Access Programs}

\begin{center}
\begin{tabular}{p{4cm} p{4cm} p{4cm} p{2cm}}
\hline
\textbf{Name} & \textbf{In} & \textbf{Out} & \textbf{Exceptions} \\
\hline
save\_model & model: Model, path: String & $\mathbb{B}$ & IOError \\
\hline
load\_model & path: String &  Model & IOError,

FormatError \\
\hline
save\_emds & embeddings: [Embedding],

path: String &  $\mathbb{B}$ & IOError \\
\hline
load\_emds & path: String &  [Embedding] & IOError \\
\hline


save\_training\_history & history: Dictionary, path: String & $\mathbb{B}$ & IOError \\
\hline
load\_training\_history & path: String & Dictionary & IOError, 

FormatError \\
\hline
\end{tabular}
\end{center}

\subsection{Semantics}

\subsubsection{State Variables}
None

\subsubsection{Environment Variables}

FileSystem: The file system where models and embeddings are stored

\subsubsection{Assumptions}

\begin{itemize}
  \item The file system is accessible and has sufficient space
  \item The paths provided are valid
\end{itemize}

\subsubsection{Access Routine Semantics}

\noindent \texttt{save\_model(model, path)}:
\begin{itemize}
\item output: True if success, False otherwise
\item exception: IOError if file cannot be written
\end{itemize}

\noindent \texttt{load\_model(path)}:
\begin{itemize}
\item output: Model
\item exception: IOError if file cannot be read, FormatError if file format is invalid
\end{itemize}

\noindent \texttt{save\_emds(embeddings, path)}:
\begin{itemize}
\item output: True if success, False otherwise
\item exception: IOError if file cannot be written
\end{itemize}

\noindent \texttt{load\_embeddings(path)}:
\begin{itemize}
\item output: [Embedding]
\item exception: IOError if file cannot be read, FormatError if file format is invalid
\end{itemize}

\noindent \texttt{save\_training\_history(history, path)}:
\begin{itemize}
\item output: True if success, False otherwise
\item exception: IOError if file cannot be written
\end{itemize}

\noindent \texttt{load\_training\_history(path)}:
\begin{itemize}
\item output: Dictionary where keys are strings representing metrics ('loss', 'accuracy', etc.) and values are lists of corresponding numeric values for each training epoch
\item exception: IOError if file cannot be read, FormatError if file format is invalid
\end{itemize}

\newpage

\section{MIS of Data Processing Module} \label{ModuleDP}

\subsection{Module}

DataProcessor

\subsection{Uses}
SystemInterface

\subsection{Syntax}

\subsubsection{Exported Constants}
None
\subsubsection{Exported Access Programs}

\begin{center}
  \begin{tabular}{p{4cm} p{4cm} p{4cm} p{2cm}}
  \hline
  \textbf{Name} & \textbf{In} & \textbf{Out} & \textbf{Exceptions} \\
  \hline
  load\_data & path: String &  DataFrame & IOError,
  
  FormatError\\
  \hline
  validate\_data & data: DataFrame &  $\mathbb{B}$ & -\\
  \hline
  prep\_data & data: DataFrame &  DataFrame & - \\
  \hline
  split\_data & data: DataFrame,
  
  train\_ratio: $\mathbb{R}$ &  (DataFrame, DataFrame) & ValueError \\
  \hline
  create\_training\_data & data: DataFrame & Dictionary & ValueError \\
  \hline
  get\_book\_mapping & data: DataFrame & Dictionary & - \\
  \hline
  \end{tabular}
  \end{center}


\subsection{Semantics}

\subsubsection{State Variables}
None

\subsubsection{Environment Variables}

None

\subsubsection{Assumptions}

\begin{itemize}
  \item Input data follows the expected schema
\end{itemize}
\subsubsection{Access Routine Semantics}

\noindent \texttt{load\_data(path)}:
\begin{itemize}
\item output: DataFrame containing the data from the file at the specified path
\item exception: \texttt{IOError} if the file cannot be read; \texttt{FormatError} if the file format is invalid
\end{itemize}

\noindent \texttt{validate\_data(data)}:
\begin{itemize}
\item output: True if the data meets all validation criteria, False otherwise
\end{itemize}

\noindent \texttt{preprocess\_data(data)}:
\begin{itemize}
\item output: DataFrame containing the input data after applying normalization and feature creation
\end{itemize}

\noindent \texttt{split\_data(data, train\_ratio)}:
\begin{itemize}
\item output: (DataFrame, DataFrame) representing training data and testing data
\item exception: \texttt{ValueError} if \texttt{train\_ratio} is not in $(0,1)$
\end{itemize}

\noindent \texttt{create\_training\_data(data)}:
\begin{itemize}
\item output: Dictionary where keys are strings 'user\_ids', 'item\_ids', 'ratings', 'user\_features', and 'item\_features', with values being arrays of user identifiers, arrays of item identifiers, arrays of numerical rating values, lists of User Feature dictionaries, and lists of Item Feature dictionaries, respectively
\item exception: \texttt{ValueError} if required features are missing
\end{itemize}

\noindent \texttt{get\_book\_mapping(data)}:
\begin{itemize}
\item output: Dictionary where each key is a book ID ($\mathbb{Z}$) and the value is a tuple of (String, String, $\mathbb{Z}$, String) representing the associated title, author, year, and publisher
\end{itemize}

\newpage

\section{MIS of Model Training Module} \label{ModuleMT}

\subsection{Module}

ModelTrainer

\subsection{Uses}
DataProcessor, NeuralNetworkArchitecture, VectorOperations

\subsection{Syntax}

\subsubsection{Exported Constants}
DEFAULT\_LEARNING\_RATE = 0.001\\
DEFAULT\_BATCH\_SIZE = 64\\
DEFAULT\_REGULARIZATION = 0.0001
\subsubsection{Exported Access Programs}

\begin{center}
  \begin{tabular}{p{3cm} p{4cm} p{4cm} p{2cm}}
  \hline
  \textbf{Name} & \textbf{In} & \textbf{Out} & \textbf{Exceptions} \\
  \hline
  initialize & config: TrainingConfig & - & ValueError \\
  \hline
  train & train\_data: Dataset,
  
  epochs: $\mathbb{Z}$
  
  & Dictionary & RuntimeError \\
  \hline
  evaluate & test\_data: Dataset
  
  & EvaluationMetrics & RuntimeError \\
  \hline
  get\_user\_model & 
  
  - &  Model & RuntimeError \\
  \hline
  get\_item\_model & 
  
  - &  Model & RuntimeError \\
  \hline
  \end{tabular}
  \end{center}

\subsection{Semantics}

\subsubsection{State Variables}
\begin{itemize}
  \item UserModel: The neural network model for the user
  \item ItemModel: The neural network model for the item 
  \item IsInitialized: Boolean indicating if the module has been initialized
  \item Config: Training configuration parameters
  \item Optimizer: Optimization algorithm
  \item Device: Computation device (CPU/GPU)
\end{itemize}

\subsubsection{Environment Variables}

None

\subsubsection{Assumptions}

\begin{itemize}
  \item The training data is preprocessed and valid
  \item The model configuration is valid
\end{itemize}

\subsubsection{Access Routine Semantics}

\noindent \texttt{initialize(config)}:
\begin{itemize}
\item transition:
\begin{itemize}
  \item \texttt{UserModel} $\leftarrow$ \texttt{config[`user\_architecture']}
  \item \texttt{ItemModel} $\leftarrow$ \texttt{config[`item\_architecture']}
  \item \texttt{Optimizer} $\leftarrow$ initialize optimization algorithm
  \item \texttt{IsInitialized} $\leftarrow$ \texttt{True}
\end{itemize}
\item exception: \texttt{ValueError} if \texttt{config} contains invalid parameters
\end{itemize}

\noindent \texttt{train(train\_data, epochs)}:
\begin{itemize}
\item transition:
\begin{itemize}
  \item Use \texttt{Optimizer} to optimize the loss function of the user and item models
  \item The loss is computed using the local function \texttt{compute\_loss}
\end{itemize}
\item output: Dictionary where keys are strings `loss', `training\_loss', and `validation\_loss', and values are list of real number.
\item exception: \texttt{RuntimeError} if \texttt{IsInitialized} is \texttt{false}
\end{itemize}

\noindent \texttt{evaluate(test\_data)}:
\begin{itemize}
\item output: EvaluationMetrics computed on the test data
\item exception: \texttt{RuntimeError} if \texttt{IsInitialized} is \texttt{false}
\end{itemize}

\noindent \texttt{get\_user\_model()}:
\begin{itemize}
\item output: Model
\item exception: \texttt{RuntimeError} if \texttt{IsInitialized} is \texttt{false}
\end{itemize}

\noindent \texttt{get\_item\_model()}:
\begin{itemize}
\item output: Model
\item exception: \texttt{RuntimeError} if \texttt{IsInitialized} is \texttt{false}
\end{itemize}

\subsubsection{Local Functions}

\noindent \texttt{compute\_loss(user\_embeddings, item\_embeddings, ratings)}:
\begin{itemize}
\item Type: $[\mathbb{R}^k] \times [\mathbb{R}^k] \times [\mathbb{R}] \rightarrow \mathbb{R}$
\item Description: Computes the mean squared error (MSE) loss between predicted and actual ratings
\end{itemize}

\newpage

\section{MIS of Embedding Generation Module} \label{ModuleEG}

\subsection{Module}

EmbeddingGenerator

\subsection{Uses}
NeuralNetworkArchitecture, VectorOperations

\subsection{Syntax}

\subsubsection{Exported Constants}
None
\subsubsection{Exported Access Programs}

\begin{center}
  \begin{tabular}{p{5cm} p{4cm} p{4cm} p{3cm}}
  \hline
  \textbf{Name} & \textbf{In} & \textbf{Out} & \textbf{Exceptions} \\
  \hline
  initialize & user\_model: Model,item\_model: Model & - & ValueError \\
  \hline
  generate\_user\_embedding & users: [User Feature] & [Embedding] & RuntimeError\\
  \hline
  generate\_item\_embedding & items: [Item Feature] & [Embedding] & RuntimeError\\
  \hline
  \end{tabular}
\end{center}

\subsection{Semantics}

\subsubsection{State Variables}
\begin{itemize}
  \item UserModel: The neural network model for the user tower
  \item ItemModel: The neural network model for the item tower
  \item IsInitialized: Boolean indicating if the module has been initialized
  \item Device: Computation device (CPU/GPU)
\end{itemize}

\subsubsection{Environment Variables}

None

\subsubsection{Assumptions}

\begin{itemize}
  \item The models have been trained
  \item User and item inputs have same dimensions
\end{itemize}
\subsubsection{Access Routine Semantics}

\noindent \texttt{initialize(user\_model, item\_model)}:
\begin{itemize}
\item transition:
  \begin{itemize}
    \item \texttt{UserModel} $\leftarrow$ \texttt{user\_model}
    \item \texttt{ItemModel} $\leftarrow$ \texttt{item\_model}
    \item \texttt{IsInitialized} $\leftarrow$ \texttt{true}
    \item \texttt{Device} $\leftarrow$ detected available hardware (CPU or GPU)
  \end{itemize}
\item exception: \texttt{ValueError} if the models are incompatible
\end{itemize}

\noindent \texttt{generate\_user\_embedding(users)}:
\begin{itemize}
\item output: [Embedding] for the provided user(s)
\item exception: \texttt{RuntimeError} if \texttt{IsInitialized} is \texttt{false}
\end{itemize}

\noindent \texttt{generate\_item\_embedding(items)}:
\begin{itemize}
\item output: [Embedding] for the provided item(s)
\item exception: \texttt{RuntimeError} if \texttt{IsInitialized} is \texttt{false}
\end{itemize}


\newpage

\section{MIS of Recommendation Module } \label{ModuleR}

\subsection{Module}

Recommender

\subsection{Uses}
EmbeddingGenerator, ANNSearch, VectorOperations

\subsection{Syntax}

\subsubsection{Exported Constants}
DEFAULT\_NUM\_RECOMMENDATIONS = 10\\
SIMILARITY\_THRESHOLD = 0.5
\subsubsection{Exported Access Programs}

\begin{center}
  \begin{tabular}{p{5cm} p{4cm} p{4cm} p{2cm}}
  \hline
  \textbf{Name} & \textbf{In} & \textbf{Out} & \textbf{Exceptions} \\
  \hline
  initialize & ann\_index: ANNIndex,
  
  embedding\_generator: EmbeddingGenerator,
  
  book\_lookup: Dictionary & -& ValueError \\
  \hline
  get\_recommendations & user: User Feature,
  
  num\_results: $\mathbb{Z}$ & [($\mathbb{Z}$, String, $\mathbb{R}$)] & RuntimeError \\
  \hline
  \end{tabular}
  \end{center}

\subsection{Semantics}

\subsubsection{State Variables}
ANNIndex: The index for approximate nearest neighbor search\\
EmbeddingGenerator: Reference to the embedding generator\\
BookLookup: Dictionary mapping item IDs to book details\\
IsInitialized: Boolean indicating if the module has been initialized

\subsubsection{Environment Variables}

None

\subsubsection{Assumptions}

\begin{itemize}
  \item The ANN index has been built with item embeddings
  \item The embedding generator has been initialized with trained models
  \item The book lookup dictionary contains valid mappings
\end{itemize}

\subsubsection{Access Routine Semantics}

\noindent \texttt{initialize(ann\_index, embedding\_generator, book\_lookup)}:
\begin{itemize}
  \item transition:
  \begin{itemize}
    \item \texttt{ANNIndex} $\leftarrow$ \texttt{ann\_index}
    \item \texttt{EmbeddingGenerator} $\leftarrow$ \texttt{embedding\_generator}
    \item \texttt{BookLookup} $\leftarrow$ \texttt{book\_lookup}
    \item \texttt{IsInitialized} $\leftarrow$ \texttt{true}
  \end{itemize}
  \item exception: \texttt{ValueError} if any parameter is invalid
\end{itemize}

\noindent \texttt{get\_recommendations(user, num\_results)}:
\begin{itemize}
  \item output: [($\mathbb{Z}$, String, $\mathbb{R}$)] representing a list of (item\_id, item\_title, similarity\_score) tuples
  \item exception: \texttt{RuntimeError} if \texttt{IsInitialized} is \texttt{false}
\end{itemize}

\subsubsection{Local Functions}

\noindent \texttt{rank\_candidates(user\_embedding, candidate\_embeddings)}:
\begin{itemize}
  \item Type: $\mathbb{R}^k \times [\mathbb{R}^k] \rightarrow [(\mathbb{Z}, \mathbb{R})]$
  \item Description: Ranks candidate items based on similarity scores computed using the dot product; returns a list of item indices with associated scores.
\end{itemize}


\newpage

\section{MIS of Neural Network Architecture Module} \label{ModuleNNA}

\subsection{Module}

NeuralNetworkArchitecture

\subsection{Uses}
VectorOperations

\subsection{Syntax}

\subsubsection{Exported Constants}
DEFAULT\_HIDDEN\_LAYERS = [256, 128]\\
DEFAULT\_ACTIVATION = "relu"
\subsubsection{Exported Access Programs}

\begin{center}
  \begin{tabular}{p{4cm} p{4cm} p{4cm} p{2cm}}
  \hline
  \textbf{Name} & \textbf{In} & \textbf{Out} & \textbf{Exceptions} \\
  \hline
  create\_user\_tower & input\_dim: $\mathbb{Z}$ ,
  
  hidden\_layers: $[\mathbb{Z}]$ ,
  
  embedding\_dim: $\mathbb{Z}$ &  Model & ValueError \\
  \hline
  create\_item\_tower & input\_dim: $\mathbb{Z}$ ,
  
  hidden\_layers: $[\mathbb{Z}]$ ,
  
  embedding\_dim: $\mathbb{Z}$ &  Model & ValueError \\
  \hline
  \end{tabular}
  \end{center}

\subsection{Semantics}

\subsubsection{State Variables}
None

\subsubsection{Environment Variables}

None

\subsubsection{Assumptions}

\begin{itemize}
  \item Input dimensions are valid positive integers
  \item Hidden layers and embedding dimensions are compatible
\end{itemize}

\subsubsection{Access Routine Semantics}

\noindent create\_user\_tower(input\_dim, hidden\_layers, embedding\_dim):
\begin{itemize}
\item output: Model for user tower
\item exception: ValueError if dimensions are invalid
\end{itemize}

\noindent create\_item\_tower(input\_dim, hidden\_layers, embedding\_dim):
\begin{itemize}
\item output: Model for item tower
\item exception: ValueError if dimensions are invalid
\end{itemize}

\newpage

\section{MIS of ANN Search Module} \label{ModuleANN}

\subsection{Module}

ANNSearch

\subsection{Uses}
VectorOperations

\subsection{Syntax}

\subsubsection{Exported Constants}
DEFAULT\_SEARCH\_NPROBE := 10\\
DEFAULT\_INDEX\_TYPE := "Flat"
\subsubsection{Exported Access Programs}

\begin{center}
  \begin{tabular}{p{4cm} p{4cm} p{4cm} p{3cm}}
  \hline
  \textbf{Name} & \textbf{In} & \textbf{Out} & \textbf{Exceptions} \\
  \hline
  build\_index & embeddings: [Embedding] ,
  
  item\_ids: $[\mathbb{Z}]$,
  
  index\_type: String & ANNIndex & ValueError \\
  \hline
  two\_stage\_search & index: ANNIndex,
  
  query: Embedding,
  
  candidates: $\mathbb{Z}$,
  
  final\_k: $\mathbb{Z}$
   & $[(\mathbb{Z}, \mathbb{R})]$ & ValueError \\
  \hline
  save\_index & index: ANNIndex, 
  
  path: String & $\mathbb{B}$ & IOError \\
  \hline
  load\_index & path: String 
   & ANNIndex & IOError, 
   
   FormatError \\
  \hline
  \end{tabular}
  \end{center}

\subsection{Semantics}

\subsubsection{State Variables}
None

\subsubsection{Environment Variables}

None

\subsubsection{Assumptions}

\begin{itemize}
  \item Embeddings are of consistent dimension
  \item Query vector is of same dimension as indexed vectors
  \item FAISS library is available
\end{itemize}

\subsubsection{Access Routine Semantics}

\noindent \texttt{build\_index(embeddings, item\_ids, index\_type)}:
\begin{itemize}
  \item output: ANNIndex
  \item exception: \texttt{ValueError} if any parameter is invalid
\end{itemize}

\noindent \texttt{two\_stage\_search(index, query, candidates, final\_k)}:
\begin{itemize}
  \item output: $[(\mathbb{Z}, \mathbb{R})]$ representing a list of (item\_id, similarity\_score) pairs for the top-$k$ nearest neighbors
  \item exception: \texttt{ValueError} if any parameter is invalid
\end{itemize}

\noindent \texttt{save\_index(index, path)}:
\begin{itemize}
  \item output: True if success, False otherwise
  \item exception: \texttt{IOError} if the file cannot be written
\end{itemize}

\noindent \texttt{load\_index(path)}:
\begin{itemize}
  \item output: ANNIndex
  \item exception: \texttt{IOError} if the file cannot be read; \texttt{FormatError} if the file format is invalid
\end{itemize}

\newpage

\section{MIS of Vector Operations Module} \label{ModuleVO}

\subsection{Module}

VectorOperations

\subsection{Uses}
None

\subsection{Syntax}

\subsubsection{Exported Constants}
EPSILON := 1e-8 (small value to prevent division by zero)
\subsubsection{Exported Access Programs}

\begin{center}
\begin{tabular}{p{2cm} p{4cm} p{4cm} p{4cm}}
\hline
\textbf{Name} & \textbf{In} & \textbf{Out} & \textbf{Exceptions} \\
\hline
dot\_product & v1: $[\mathbb{R}]$, v2: $[\mathbb{R}]$ & $\mathbb{R}$ & DimensionMismatchError \\

\hline
\end{tabular}
\end{center}

\subsection{Semantics}

\subsubsection{State Variables}
None

\subsubsection{Environment Variables}

None

\subsubsection{Assumptions}

None

\subsubsection{Access Routine Semantics}

\noindent \texttt{dot\_product(v1, v2)}:
\begin{itemize}
  \item output: $\mathbb{R}$ representing $\sum_{i=1}^{n} v1_i \cdot v2_i$, where $n$ is the number of elements in each vector
  \item exception: \texttt{DimensionMismatchError} if the input vectors do not have the same number of elements
\end{itemize}


\newpage

\bibliographystyle {plainnat}
\bibliography {../../../refs/References}

\newpage

\newpage{}

\end{document}